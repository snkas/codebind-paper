\section{Which networking papers can benefit from \sysname?}\vspace{4pt}

We examined all $65$ NSDI-2020 papers to assess which ones could benefit from \sysname. A tabulation with greater detail is available in Appendix~\ref{appendix:nsdianalysis}, but in summary:

\begin{itemize}[leftmargin=10pt,itemsep=2pt,topsep=2pt]

    \item $38$ papers ($58\%$) have parameters or experimental settings that could be interesting for readers to tweak.
    
    \item $23$ papers ($35\%$) make use of general compute to perform computation, simulation or emulation experiments. Of these, \textbf{14 papers (22\%)} have parameters or experimental settings that could be interesting for readers to tweak.
    
    \item $19$ papers ($29\%$) make use public cloud infrastructure to run system deployment experiments. Of these, \textbf{17 papers (26\%)} have parameters readers may want to tinker with.
    
    \item \textbf{5 papers (8\%)} have a significant focus on a data set, which \sysname can help expose to readers in a more engaging, interactive manner.
    
\end{itemize}

\noindent In all, \textbf{18 papers (28\%)} could directly benefit from the features of \sysname, engaging readers in modifying experiment settings or/and in exploring substantial datasets. If authors automate cloud experiments and readers have access to cloud infrastructure to run such experiments, 30 papers (46\%) could benefit. Further, even for the work that depends on bespoke testbeds, \sysname-enabled papers allow readers to dive deeper into the experimental results.
