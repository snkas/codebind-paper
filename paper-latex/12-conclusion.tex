\section{Conclusion}
\label{sec:conclusion}

We introduced \sysname, a system for authors to: (a) expose configurable parts of their experiments to readers in an easy-to-use manner; and (b) enforce consistency between their paper and its underlying experiments. With the addition of simple and minimally obtrusive mark-up to its \LaTeX{} source, the paper's text itself controls the experiments. This enables readers to interact with the text and understand the behavior of the involved systems under different inputs, \textit{without} needing any knowledge of the underlying system code. 

We showed \sysname in action across three types of experiments: using \textit{congestion control} and \textit{data center} experiments, we showed \sysname's utility in allowing readers to tweak the large number of parameters and knobs in such experiments; and using a \textit{dataset analysis} example, we showed that even when readers cannot rerun some experiments, they can still benefit from unprecedented interactivity with the data presented in the paper. We also showed \sysname's utility in the teaching of networking concepts, with an illustration on \textit{max-min fairness} that allows students to learn the concept deeply by interacting with the example.

This paper is itself written using \sysname. A video demo of \sysname, using parts of this paper, is \href{https://drive.google.com/file/d/1U_3MxQ84FaE8fkTME6MF_SFuZIRhzTnv/view}{available online}. The code base for this paper is linked on page 1, and \textbf{we encourage readers to themselves modify the \LaTeX{} code in \S\ref{sec:example}, \S\ref{sec:congestion-control}, \S\ref{sec:netload}, and \S\ref{sec:dataset-analysis} to see how the text and results change in place.} \sysname presents a way forward for authors to make their work more consistent, reproducible, and transparent to readers.\footnote{Another potential benefit of \sysname is in accessibility: for readers who are visually-impaired, having a system like \sysname could enable them to interactively explore texts with the aid of text-to-speech tools. This can reduce the barrier they might face otherwise in extracting information from the graphical elements that typically summarize results in papers. However, this is not something we have been able to get feedback on yet.}
