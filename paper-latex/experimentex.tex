%%%%%%%%%%%%%%%%%
% EXPERIMENTEX
%
% First, the five primary commands are defined.
%
% The ExperimenTeX commands are first interpreted and
% turned into runs such that afterwards the missing
% output plot files are present for the real LaTeX compilation.
% If the interpreting and plotting have not taken place,
% the LaTeX compilation will simply use placeholders.
%
% Second, two secondary commands related to checksums are defined.
%

% Package dependencies
\usepackage{color}
\usepackage{graphics}
\usepackage{graphicx}

%%%%%%%%%%%%%%%%%%%%%%%%%%%%%%%%%%%%%%%%%%%%%%%%%%%%%%%%%%%%
%
% Format:       \expclass{name-sub}{name-super}
%
% Description:  Declare the existence of class [name-sub], which inherits
%               from class [name-super]. This inheritance means it is of the same
%               experiment type, and inherits all its explines.
%
%               In the rendered document, nothing is shown.
%
\newcommand{\expclass}[2]{}

%%%%%%%%%%%%%%%%%%%%%%%%%%%%%%%%%%%%%%%%%%%%%%%%%%%%%%%%%%%%
%
% Format:       \expinstance{name-inst}{name-super}
%
% Description:  Declare the existence of instance [name-inst], which inherits
%               from class [name-super]. This inheritance means it is of the same
%               experiment type, and inherits all its explines.
%
%               In the rendered document, nothing is shown.
%
\newcommand{\expinstance}[2]{}

%%%%%%%%%%%%%%%%%%%%%%%%%%%%%%%%%%%%%%%%%%%%%%%%%%%%%%%%%%%%
%
% Format:       \expline{name}{expline}
%
% Description:  Add [expline] to the list of explines of instance/class [name].
%
%               In the rendered document, it simply shows [expline] like any other text.
%
\newcommand{\expline}[3][]{{#3}}

%%%%%%%%%%%%%%%%%%%%%%%%%%%%%%%%%%%%%%%%%%%%%%%%%%%%%%%%%%%%
%
% Format:       \expincludegraphics[ic-options]{exp-name}{output-file.ext}
%
% Description:  Add output.ext to the list of expinclude filenames of instance [name-inst].
%
%               In the rendered document, it behaves like \includegraphics.
%               If the output file does not exist (yet), a placeholder image is depicted.
%
\newcommand{\expincludegraphics}[3][]{%
\IfFileExists{../temp/plots/#2/#3}{%
\includegraphics[#1]{../temp/plots/#2/#3}%
}{%
\includegraphics[#1]{example-image-a}%
}%
}

%%%%%%%%%%%%%%%%%%%%%%%%%%%%%%%%%%%%%%%%%%%%%%%%%%%%%%%%%%%%
%
% Format:       \expincludetext{exp-name}{output-file.ext}
%
% Description:  Add output.ext to the list of expinclude filenames of instance [name-inst].
%
%               In the rendered document, it behaves like \input.
%               If the output file does not exist (yet), a placeholder red [X] is put.
%
\newcommand{\expincludetext}[2]{\IfFileExists{../temp/plots/#1/#2}{\input{../temp/plots/#1/#2}\unskip}{\textbf{\textcolor{red}{[\texttt{X}]}}}}

%%%%%%%%%%%%%%%%%%%%%%%%%%%%%%%%%%%%%%%%%%%%%%%%%%%%%%%%%%%%
%
% Format:       \expcodecksum{}
%
% Description:  Code checksum (git commit hash or archive SHA-256)
%               It is imported from file code-checksum.txt
%
\newcommand{\expcodechecksum}[0]{\IfFileExists{code-checksum.txt}{\input{code-checksum.txt}\unskip}{\textbf{\textcolor{red}{[\texttt{not found: code-checksum.txt}]}}}}
